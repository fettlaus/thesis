Grobstruktur der Arbeit
\begin{itemize}
  \item Einführung
  \begin{itemize}
    \item Motivation - Eingebettete Systeme behrrschen unseren Alltag. Funk ist
    allgegenwärtig in Heim und Industrie
    \item Ziele - Eine Möglichkeit ein System zu planen, entwerfen und
    umzusetzen.
    \item Gliederung - Wie diese Arbeit aufgebaut ist
  \end{itemize}
  \item Analyse
  \begin{itemize}
    \item Grundidee - wie das System aufgebaut werden soll. (Bild)
    \item Bestandteile einer (allgemeinen) Entwicklungsplattform analysieren
    (Vorgehensmodell, Software, Hardware)
    \item Vorgaben - Funk, ARM, linuxbasiert
    \item Vorentscheidungen - Olimex ARM, Olimex-Funk, Carambola
    \item Anforderungen - Welche Punkte erfüllt werden müssen: Debugging,
    Analyse von zentral gesammelten (Funk-)Daten, Deployment einer Anwendung
  \end{itemize}
  \item Konzeption
  \begin{itemize}
    \item Debugging
    \begin{itemize}
      \item Was Debugging bedeutet, welche Bestandteile benötigt werden
      \item Was existiert - OpenOCD, ZY1000, sonstige
      Debugger? => Anpassung von OpenOCD an OpenWRT
      \item Cross-Compilierung
      \item OpenWRT Paketverwaltung - 
      \item Wahl eines JTAG Adapters
      \item Zielsetzung für das Debugging
  	\end{itemize}
  	\item Datenanalyse
  	\begin{itemize}
  	  \item Anforderungen - Zusammenführung von Daten, Zeitlicher Ablauf
  	  erkennbar
  	  \item Bestandteile - Client, Server, Protokoll um beides zu verbinden
  	  \item Zeitsynchronisation - NTP, Ablaufdiagramm
      \item Protokoll - Tabelle, Erläuterung
    \end{itemize}
    \item Deployment
    \begin{itemize}
      \item Flashen einer Firmware
      \item Nutzung von JTAG
    \end{itemize}
  \end{itemize}
  \item Realisierung
  \begin{itemize}
    \item OpenOCD
    \item Server(C++-Programm)
    \item Client(Java-Programm)
    \item Deployment(Skript)
  \end{itemize}
  \item Ausblick/Erweiterungsmöglichkeiten
\end{itemize}
\setcounter{page}{0}