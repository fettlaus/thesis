\chapter{Einführung}
\adjustmtc
\minitoc
\section{Motivation}
Eingebettete Systeme sind aus unseren heutigen Welt nicht mehr wegzudenken. Sie
dienen zur Heimautomatisierung, Prozesssteuerung und vielem
Anderen\footnote{weiteres Beispiel einfügen!}. Diese Systeme kommunizieren
häufig über Funkprotokolle, die auf eine hohe Energieeffizienz ausgelegt sind
und sich oftmals dynamisch organisieren können. Als Beispiel für diese
Funksysteme lassen sich ZigBee, ANT, Z-Wave, Bluetooth low energy oder MiWi
nennen.

Gleichzeitig sind diese Systeme stark in bestehende Umgebungen wie Anlagen oder
Bausubstanzen integriert beziehungsweise müssen in eine solche Umgebung
integriert werden.

Die Problematik liegt hierbei in der Entwicklung dieser vernetzten Systeme. Soll
ein Funknetzwerk integriert werden, offenbaren sich manche Schwierigkeiten erst
in der Integrationsphase und auch eine fortlaufende Wartung dieser Systeme ist,
aufgrund der teilweise hohen räumlichen Abstände, nicht immer sehr einfach du
handhaben.

Es wäre also wünschenswert, die Entwicklung, 

All diesen Funkprotokollen ist gemein, dass sie auf eine hohe Integration
ausgelegt sind und oftmals in bestehende Systeme eingebunden werden. So ist es
zum nicht abwegig, dass zusätzlich zu einer Heimautomatisierung ein
Funknetzwerk nach IEEE 802.11 \footnote{Bekannt als "`WLAN"'} bereits
besteht.

\cite*{sample_bib}
\section{Ziele}
Ziel dieser Bachelorarbeit ist es, eine Entwicklungsplattform zu schaffen, die
eine (Fort-)Entwicklung von bereits integrierten Funksystemen erleichtert.
Hierzu werden verschiedene Hilfsmittel benötigt.

Als Anforderung werden folgende Ziele gesetzt:
\begin{itemize}
  \item Möglichkeit eine Anzahl von Zielsystemen zu programmieren und zu
  debuggen.
  \item Aufnahme und Weiterleitung von Nachrichten mehrerer Zielsysteme. Hierbei
  ist insbesondere von Bedeutung, die Funkaktivitäten der Komponenten sichtbar
  zu machen.
  \item Aggregierung der empfangenen Daten mehrerer Zielsysteme in einem
  Entwicklungssystem. Die Darstellung der Daten soll einen zeitlichen Ablauf
  der jeweiligen Vorgänge ersichtlich werden lassen.
\end{itemize}
\section{Gliederung}
Im ersten Teil werden bestehende Systeme analysiert, um 
