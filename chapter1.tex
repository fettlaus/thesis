\chapter{Einführung}
\adjustmtc
\minitoc
\section{Motivation}
Eingebettete Systeme sind aus unseren heutigen Welt nicht mehr wegzudenken. Sie
dienen zur Heimautomatisierung, Prozesssteuerung und vielem
Anderen. Diese Systeme kommunizieren
häufig über Funkprotokolle, die auf eine hohe Energieeffizienz ausgelegt sind
und sich oftmals dynamisch organisieren können. Als Beispiel für diese
Funksysteme lassen sich ZigBee, ANT, Z-Wave, Bluetooth low energy oder MiWi
nennen.

Gleichzeitig sind diese Systeme stark in bestehende Umgebungen wie Anlagen oder
Bausubstanz integriert beziehungsweise müssen in eine solche Umgebung
integriert werden.

Die Problematik liegt hierbei in der Entwicklung dieser vernetzten Systeme. Soll
ein Funknetzwerk integriert werden, offenbaren sich manche Schwierigkeiten erst
in der Integrationsphase und auch eine fortlaufende Wartung dieser Systeme ist,
aufgrund der teilweise hohen räumlichen Abstände, nicht immer sehr einfach zu
handhaben.

Es wäre also wünschenswert, die Entwicklung,  <\ldots>

\section{Ziele}
Ziel dieser Bachelorarbeit ist es, eine Entwicklungsplattform zu schaffen, die
eine (Fort-)Entwicklung von bereits integrierten Funksystemen erleichtert.
Hierzu werden verschiedene Hilfsmittel benötigt.

Die einzelnen Bestandteile eines Entwicklungssystems müssen herausgearbeitet
entworfen und implementiert werden

\section{Gliederung}
Im ersten Teil werden bestehende Systeme analysiert, um <\ldots>
