\chapter{Einführung}
\adjustmtc
\minitoc
\section{Motivation}
Eingebettete Systeme sind aus unseren heutigen Welt nicht mehr wegzudenken. Sie
dienen zur Heimautomatisierung, Prozesssteuerung und vielem
Anderen. Diese Systeme kommunizieren
häufig über Funkprotokolle, die auf eine hohe Energieeffizienz ausgelegt sind
und sich oftmals dynamisch organisieren können. Als Beispiel für diese
Funksysteme lassen sich \emph{ZigBee}, \emph{ANT}, \emph{Z-Wave},
\emph{Bluetooth low energy} oder \emph{MiWi} nennen.

Gleichzeitig sind diese Systeme stark in bestehende Umgebungen wie Anlagen oder
Bausubstanz integriert beziehungsweise müssen in eine solche Umgebung
integriert werden.

Die Problematik liegt hierbei in der Entwicklung dieser vernetzten Systeme. Wird
ein einzelnes System entwickelt, sind die Randbedingungen meist recht klar
definiert und gängige Testverfahren können die Einhaltung der Vorgaben
überprüfen.

Soll jedoch ein Funknetzwerk, bestehend aus mehreren einzelnen Systemen,
entwickelt werden, offenbaren sich manche Schwierigkeiten erst im
Zusammenspiel der einzelnen Komponenten. Auch können störende
Umwelteinflüsse in der Konzeption nicht unbedingt immer vollständig erfasst und
bedacht werden.

Und auch eine fortlaufende Wartung dieser Systeme ist, aufgrund der teilweise
hohen räumlichen Abstände, nicht immer sehr einfach zu handhaben. 

Es wäre also wünschenswert, die Entwicklung,  <\ldots>
\section{Existierende Lösungen}
<TinyOS, andere Ideen>
\section{Ziele}
Ziel dieser Bachelorarbeit ist es, eine Entwicklungsplattform zu schaffen,
mit deren Hilfe sich integrierte und mit einem Funksystem ausgestattete
Zielsysteme entwickeln und in ihrem Betrieb untersuchen lassen.
\begin{definition}[Zielsystem]
Das \emph{Zielsystem} ist das System, welches mit Hilfe des
\emph{Entwicklungssystems} relisiert werden soll.
\end{definition}
Hierzu werden verschiedene Hilfsmittel benötigt.

Die einzelnen Bestandteile eines Entwicklungssystems müssen herausgearbeitet,
entworfen und implementiert werden
<Genauer abgrenzen>
\section{Gliederung}
Im ersten Teil werden bestehende Systeme analysiert, um <\ldots>
