\chapter{Einführung}
\adjustmtc
\minitoc
\section{Motivation}
Eingebettete Systeme sind aus unseren heutigen Welt nicht mehr wegzudenken. Sie
dienen zur Heimautomatisierung, Prozesssteuerung und vielem
Anderen. Diese Systeme kommunizieren häufig über Funkprotokolle, die auf eine
hohe Energieeffizienz ausgelegt sind und sich in ihrem Netzwerkaufbau oftmals
dynamisch organisieren können. Als Beispiel für derartige Funksysteme lassen
sich \emph{ZigBee}, \emph{ANT}, \emph{Z-Wave}, \emph{Bluetooth low energy} oder
\emph{MiWi} nennen.

Gleichzeitig sind diese Systeme stark in bestehende Umgebungen wie Anlagen oder
Bausubstanz integriert beziehungsweise müssen in eine solche Umgebung
integriert werden.

Die Problematik liegt hierbei in der Entwicklung dieser vernetzten Systeme. Wird
ein einzelnes System entwickelt, sind die Randbedingungen meist recht klar
definiert, Anwendungen zur Softwareentwicklung sind weit verbreitet und gängige
Testverfahren können die Einhaltung der Vorgaben überprüfen.

Soll jedoch ein Funknetzwerk, bestehend aus mehreren einzelnen Systemen,
entwickelt werden, offenbaren sich manche Schwierigkeiten erst im
Zusammenspiel der einzelnen Komponenten. Auch können störende
Umwelteinflüsse in der Konzeption nicht immer vollständig erfasst und
bedacht werden.

Dazu kommt, dass meist eine Vielzahl identischer Systeme miteinander
kommuniziert und so auch oft die gleiche Software auf diesen Systemen zum
Einsatz kommt.

Und auch eine fortlaufende Wartung dieser Systeme ist, aufgrund der teilweise
hohen räumlichen Abstände, nicht immer sehr einfach zu handhaben. 

\section{Existierende Lösungen}\label{sec:exist}
Für stark in ihren Ressourcen eingeschränkte, verteilte Funksysteme existieren
eine Anzahl hierfür optimierter Betriebssysteme. Das TinyOS-Projekt\cite{TINY}
zum Beispiel ist ein an der Universität Berkeley entstandenes und spezifisch für
den Betrieb drahtloser Sensornetzwerke entwickeltes Betriebssystem. Weitere
Beispiele wären Contiki\cite{CONT} oder FreeRTOS\cite{FREE}.

Für Änderung oder Aktualisierung der meisten dieser Betriebssysteme
wird eine direkte Verbindung zu dem Rechner des Entwicklers benötigt. Dies kann,
insbesondere mit einer hohen Anzahl Sensorknoten, durchaus zu einer logistischen
Herausforderung werden.

\begin{definition}[Zielsystem]
Das \emph{Zielsystem} ist das System, welches mit Hilfe des
\emph{Entwicklungssystems} realisiert werden soll.
\end{definition}

Um dieses Problem zu umgehen, verfolgt TinyOS einen anderen Ansatz. Will man auf
mehrere TinyOS-basierte Zielsysteme beziehungsweise Sensorknoten eine neue
Softwareversion aufspielen, kann man sich die in das System integrierte Software
"`Deluge 2"'\cite{DELUG} zunutze machen. Diese Software ermöglicht es, eine neue
Softwareversion mittels eines Bootloaders und über das bereits bestehende
Funknetzwerk zu verteilen. Hierbei funktioniert die Verteilung "`epidemisch"'
und wird damit größtenteils durch die Sensorknoten selbst verwaltet.

Dieses Vorgehen hat jedoch wiederum den Nachteil, dass für eine Verteilung der
Software überhaupt erst einmal ein Netzwerk unter den Sensorknoten existieren
muss. So wird dieses Verfahren unmöglich, sobald ein komplett neues
Funkprotokoll entwickelt werden soll.

Wünschenswert wäre es also, die Verteilung neuer Softwareversionen möglichst
kabellos und gleichzeitig ohne starke Abhängigkeit vom Zielsystem oder dem
verwendeten Funksystem durchführen zu können.

Während der Entwicklung eines solchen Funksystems spielen auch andere
Aspekte eine Rolle. So ist es ab einem gewissen Punkt in der Entwicklung sicher
wichtig, diese Systeme und die Abläufe im Funknetz im laufenden Betrieb
analysieren zu können.

\section{Ziele}
Ziel dieser Bachelorarbeit ist es, ein Entwicklungssystem zu schaffen,
mit dessen Hilfe sich integrierte und mit einem Funkmodul ausgestattete
Zielsysteme entwickeln und in ihrem Betrieb untersuchen lassen.

Hierzu werden verschiedene Hilfsmittel benötigt, deren einzelne Bestandteile
herausgearbeitet, entworfen und implementiert werden sollen.
\section{Gliederung}
In \autoref{chap:analyse} "`\nameref{chap:analyse}"' werden die Elemente eines
allgemeinen Entwicklungssystems analysiert. Anschließend sollen Szenarien
vorgestellt werden, die bei der Entwicklung eines verteilten,
eingebetteten Systems auftreten können. Daraus erwachsend sollen
dann die zu erfüllenden Anforderungen an das zu entwerfende
Entwicklungssystem gestellt werden.

In \autoref{chap:konzeption} "`\nameref{chap:konzeption}"' wird zuerst die
grundlegende Idee eines verteilten Entwicklungssystems erläutert, bevor
anschließend Vorentscheidungen über die zu verwendende Hardware getroffen
werden. Hierbei müssen auch die für die Realisierung wichtigen Randbedingungen
ermittelt werden.

\autoref{chap:realisierung} "`\nameref{chap:realisierung}"' beschreibt dann die
Umsetzung des konzipierten Systems, erläutert die Funktionen der einzelnen
Bestandteile und analysiert abschließend das System hinsichtlich der Umsetzung
der Vorgaben.

\autoref{chap:ausblick} "`\nameref{chap:ausblick}"' bietet zum Schluss einen
Ausblick auf mögliche Verbesserungen und Fortentwicklungen des Systems.