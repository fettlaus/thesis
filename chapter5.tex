\chapter{Ausblick und Fazit}\label{chap:ausblick}
\minitoc
\section{Verbesserungen des entwickelten Systems}
Einige Komponenten des entwickelten Systems erfüllen zwar die gestellten
Anforderungen, könnten in ihrer Funktionsweise jedoch noch verbessert werden.
Diese Verbesserungen betreffen zum Beispiel die Genauigkeit der Zeitstempel oder
den Ressourcenverbrauch der Software.

Einige denkbare Verbesserungen sollen hier angeführt werden.
\subsection{Kompensation des Clock-Drifts}
Die Analyse aus \autoref{sect:analyse} zeigt, dass die Herstellungstoleranzen
der Schwingquarze einen starken Drift verursachen und so die Zeitmessung massiv
beeinflussen.

Die erwähnten Funktionen von \texttt{adjtimex} könnte man in der Serversoftware
in Zusammenspiel mit dem Client verwenden, um eine "`gemeinsame Taktrate"' zu
errechnen. 

Dadurch wäre es unter Umständen möglich, Betrachtungen von Zeitabläufen auch
über einen längeren Zeitraum durchführen zu können, ohne regelmäßig eine
Zeitsynchronistion durchführen zu müssen.
\subsection{Parallelisierung}
Da die in \autoref{sect:deployment} vorgestellten Skripte die Programmierung der
einzelnen Zielsysteme seriell ausführen, wird für eine zunehmende Anzahl an
Zielsystemen auch zunehmend mehr Zeit zum Flashen nötig.

Außerdem ist ein durch solch ein Skript verursachter "`gemeinsamer Neustart"'
immer signifikant zeitversetzt.

Eine denkbare Lösung hierfür wäre es, das Flashen und Neustarten
unterschiedlicher Zielsysteme von mehreren Threads gleichzeitig ausführen zu
lassen.
\subsection{Speicherverbrauch des Servers}
Im Betrieb von \emph{FreeJTAG} stellte sich heraus, dass das Programm einen
relativ hohen Speicherverbrauch hat ($\approx \SI{9400}{\kilo\byte}$). 

Dies ist für eine derartige Anwendung auf einem eingebetteten System ein
ziemlich hoher Wert. So benötigt zum Beispiel das relativ komplexe
\emph{OpenOCD} ungefähr \SI{2600}{\kilo\byte}.

Dieser Speicherverbrauch ist für das Carambola mit seinen \SI{32}{\mega\byte}
Arbeitsspeicher zwar noch keine Schwierigkeit, schränkt jedoch die Verwendung
weiterer Software ein und verhindert unter Umständen die Portierung auf
leistungsschwächere Hardware.

Eine Optimierung der Serversoftware ist also unter Umständen möglich und
eröffnet weitere Optionen, mit dem Carambola zu arbeiten.
\section{Mögliche Erweiterungen}
Da das gesamte Entwicklungssystem in Hinblick auf größtmögliche Flexibilität und
Modularisierung entwickelt wurde, sind eine Vielzahl möglicher Erweiterungen
denkbar. Einige dieser Möglichkeiten sollen hier aufgeführt werden.
\subsection{Zusätzliche Anschlussmöglichkeiten}
Während für die Kommunikation zwischen Carambola und Zielsystem zur Zeit
\gls{uart} eingesetzt wird, ist auch eine Erweiterung auf die
\gls{spi}-Schnittstelle denkbar.

Hierfür müsste in die Serversoftware ein \texttt{SPIService} integriert werden.
Durch die Entkoppelung zwischen \texttt{NetworkService} und
\texttt{UARTService} wäre es zum Beispiel möglich, den bereits vom
\texttt{UARTService} benutzten Buffer zu verwenden, um die erhaltenen
SPI-Nachrichten an den Rest des Systems weiterzuleiten.

Diese Vorgehensweise hätte den Vorteil, dass dadurch keine weiteren
Änderungen am Rest des Systems nötig wären.

Ein anderer Ansatz wäre, eine der ungenutzten IDs des Netzwerkprotokolls zu
verwenden, um \gls{spi}-Daten als separate Werte im Client erfassen zu
können. Dies erfordert jedoch etwas umfangreichere Änderungen am Server, da ein
weiterer Thread und ein weiterer Buffer angelegt werden müssten. Außerdem müsste
die Clientsoftware um das Empfangen von \gls{spi}-Datenpaketen erweitert werden.

Eine ähnliche Erweiterung wäre auch für die Weiterleitung von I$^2$C-Daten
denkbar.
\subsection{Analyse via Wireshark}
Das das Entwicklungssystem hauptsächlich auf die Analyse von Funknetzwerken
zugeschnitten ist, wäre auch diesbezüglich eine Erweiterung denkbar.

So könnte man den Client zum Beispiel die gesammelten Netzwerkdaten in eine
Datei schreiben lassen.

Da der Netzwerkanalyzer "`Wireshark"'\cite{WIRESH} umfangreiche Möglichkeiten
bietet, ihm neue Protokolle hinzuzufügen, könnte man ihn unter Umständen auch
dazu verwenden, diese Datei einzulesen.

Würde man diesen Gedanken noch weiter tragen, wäre es wünschenswert, mit
Wireshark eine direkte Verbindung zu den einzelnen Carambolas aufzubauen. So
ließe sich hierüber ein verteilter Netzwerksniffer realisieren.
\section{Fazit}
Es konnte erfolgreich ein System entwickelt werden, mit dessen Hilfe die
Entwicklung eines verteilten Funknetzwerks eingebetteter Systems wesentlich
erleicht wird.

Wird eine Software für ein solches eingebettetes System entwickelt, ist es
nun möglich, fast alle Entwicklungsphasen netzwerkgestützt durchzuführen.

Dazu besteht das gesamte Entwicklungssystem aus einer Anzahl von
OpenWRT-Plattformen, von denen jeweils eine einem Zielsystem
zugeordnet ist. Diese Plattform ist durch eine direkte Verbindung in der
Lage, physischen Zugriff auf die \gls{jtag}-Schnittstelle des Zielsystems zu
nehmen. Durch den Einsatz von OpenOCD kann ein \gls{gdb}-Client verwendet
werden, um das Zielsystem flashen und debuggen zu können.

Zusätzlich ist es dem OpenWRT-Modul mit Hilfe der entworfenen Software
"`FreeJTAG"' möglich, die von seinem Netzwerkknoten ausgegebenen
\gls{uart}-Daten zu erfassen.

Über ein WLAN oder Ethernet sind alle einzelnen OpenWRT-Module mit der
Clientsoftware "`The Kraken"' verbunden, die die über \gls{uart} erfassten Daten
sammelt und nach ihren, von "`FreeJTAG"' gesetzten, Zeitstempeln sortiert. Die
Kommunikation zwischen "`FreeJTAG"' und "`The Kraken"' erfolgt dabei über ein
selbstentwickeltes Protokoll.

Da für die Erfassung von Funkdaten üblicherweise eine hohe zeitliche Genauigkeit
benötigt wird, wurde besonderes Augenmerk auf die Zeitsynchronisation zwischen
den beiden Softwarebestandteilen gelegt. So wurde für das entworfene Protkoll
ein Synchronisationsablauf entwickelt, der in Tests eine hohe Präzision zeigte.

Allerdings war es im Zeitraum dieser Arbeit nicht möglich, ein beispielhaftes
Funksystem zu implementieren, sondern nur eine Software zur Prüfung der
Präzision des Entwicklungssystems. Dies war jedoch auch keine für die Arbeit
essentielle Aufgabe, da nur die Voraussetzungen für die Entwicklung eines
solchen Systems geschaffen werden sollten.

Die Testergebnisse zeigen außerdem, dass sowohl Debugging als auch Deployment
und Analyse funktionieren und sich somit ein solches Funksystem unter
Nutzung dieses Entwicklungssystems entwickeln ließe.

Ergänzt man die vorgeschlagenen Änderungen, könnte sich das Entwicklungssystem
als flexible, robuste und günstige Lösung zur Vereinfachung der Entwicklung
und Analyse eingebetteter Funknetzwerke erweisen.