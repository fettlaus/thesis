\chapter{Installationshinweise}
Die beigelegte CD enthält den Quellcode der im Rahmen dieser Bachelorarbeit
entwickelten Software.

Da die Serversoftware stark in das Build-System von OpenWRT integriert
ist, folgt eine Auflistung der selbstentwickelten oder modifizierten Dateien.
\vspace{1pt}
\dirtree{%
.1 Hauptordner.
.2 Thesis.pdf Installation.
.2 scripts.
.3 flash-all run-all.
.2 binary.
.3 TheKraken.exe.
.3 chibios.elf.
.3 openwrt-ramips-rt305x-carambola-squashfs-sysupgrade.bin.
.2 thekraken.
.3 src.
.4 \ldots.
.3 pom.xml.
.2 chibios.
.3 \ldots.
.2 carambola.
.3 package.
.4 freejtag.
.5 files.
.6 freejtag.uci-defaults openocd.cfg openocd.init. 
.5 src.
.6 \ldots.
.5 Makefile.
.4 openocd.
.5 Makefile.
.3 feeds/packages/libs/libftdi.
.4 Makefile.
}
\vspace{1pt}
Alle weiteren Dateien sind ausdrücklich nicht Bestandteil dieser Arbeit,
allerdings für die Verwendung der Software notwendig.
In der Datei \texttt{Installation} befinden sich Hinweise zur Konfiguration des
Entwicklersystems und zum Kompilieren eines OpenWRT Images.