\chapter{Realisierung}
In diesem Kapitel wird die Realisierung des Entwicklungssystems
beschrieben. <\ldots>
\section{Konfiguration des Carambola}
OpenWRT bietet mit dem Programm \gls{uci} eine Anwendung, Einstellungen
zentralisiert zu verwalten. Über das \gls{uci}-System lassen sich unter anderem
Ethernet, WLAN, DHCP und SSH-Server konfigurieren. Die einzelnen
Einstellungsdateien liegen alle im Verzeichnis \listinlsh{/etc/config/} und
können mittels des Tools modifiziert werden.

Nach Systemstart werden außerdem alle Dateien im Ordner
\listinlsh{/etc/uci-defaults/} eingelesen, ausgeführt und anschließend gelöscht.
Dies erlaubt es Softwarepaketen, Einstellungen am System vorzunehmen. Werden
diese Softwarepakete bereits in die Firmware integriert, entspricht dies einer
Art Vorkonfiguration des Systems.
\subsection{WLAN und Netzwerkkonfiguration}
Da die Netzwerkkonfiguration von OpenWRT vollständig durch die
\gls{uci}-Konfigurationsdateien abgewickelt wird, muss auch \gls{uci} genutzt
werden, wenn das Netzwerk vorkonfiguriert werden soll.

\gls{uci} bietet mit \listinlsh{uci batch} explizit einen Befehl an, um
umfangreiche Einstellungsändeurngen vorzunehmen. Hierfür werden die einzelnen
Befehle mittels \emph{Here document} übergeben.
\begin{lstlisting}[language=sh]
uci batch <<-EOF_network
	...
	commit network
	EOF_network
\end{lstlisting}

 \begin{definition}[Here document]
Ein \emph{Here document} ermöglicht es, einem Unix-Befehl mehrere, durch
Zeilenumbrüche getrennte, Befehle zu übergeben.
\end{definition}

\subsection{Aktivierung des UART}\label{subs:aktuart}
Der Mikrochip\cite{RA01} des Carambolas besitzt zwei \glspl{uart}. Einer dieser
\glspl{uart} wird für die Serielle Konsole verwendet und somit kann der zweite
Anschluss für die Datenschnittstelle des Entwicklungssystems verwendet werden.
Da die Pins des Carambolas für den zweiten \gls{uart} standardmäßig auf
\gls{gpio}-Betrieb eingestellt sind, müssen diese erst umgestellt werden.

Hierzu muss das Tool \texttt{io} installiert und mittels 
\listinlsh{io 0x10000060 0x01} ausgeführt werden. Dies setzt im Speicher des
Mikrocontrollers das Flag, den \gls{uart} zu aktivieren. Dieser Vorgang muss
nach jedem Systemstart erfolgen und kann durch die, bei jedem Systemstart
aufgerufene, \listinlsh{/etc/rc.local} erfolgen.

Zusätzlich muss verhindert werden, dass auf diesem \gls{uart} eine Linuxterminal
betrieben wird. Hierzu muss die Datei \listinlsh{/etc/inittab} modifiziert
werden.

Diese beiden Änderungen werden wie folgt ebenso mittels des Skripts in
\listinlsh{/etc/uci-defaults} durchgeführt.
\begin{lstlisting}[language=sh]
sed -i '/exit 0/ i\io 0x10000060 0x01' /etc/rc.local
sed -i '/ttyS0/ s/^/# /' /etc/inittab
\end{lstlisting}

\section{OpenOCD}
Für OpenOCD existiert keine Portierung in die Paketverwaltung von OpenWRT. Es
muss also ein Makefile erstellt werden, das die Einbindung ermöglicht. Der
grundlegende Aufbau solcher Makefiles ist im Wiki von OpenWRT
festgelegt\cite{OWRT}.

Um OpenOCD kompilieren zu können, müssen zuerst die Voraussetzungen festgestellt
werden. Da ein auf dem FT2232-Chip basierender \gls{jtag}-Adapter mit
USB-Anschluss eingesetzt werden soll, müssen laut \texttt{README} von OpenOCD
sowohl libftdi als auch libusb installiert sein. Diese Pakete stehen für OpenWRT
bereits zur Verfügung und müssen im Makefile als Abhängigkeiten definiert
werden. Dies geschieht mit dem Befehl \listinlsh{DEPENDS:=+libftdi +libusb}.

Der Quellcode von OpenOCD wird durch das Makefile in der Version 0.6.1 von
Sourceforge selbstständig heruntergeladen, über MD5 verifiziert und entpackt.

Vor dem Kompilierungsvorgang müssen noch die an \listinlsh{./configure}
zu übergebenden Argumente festgelegt werden. Der verwendete Adapter erfordert
hier die Option \listinlsh{--enable-ft2232_libftdi}.

<\ldots>
<Vorkonfiguration von OpenWRT/OpenOCD>
\section{Server - FreeJTAG}
Der "`FreeJTAG"' genannte Server des Entwicklungssystems wird auf dem Carambola
installiert, als Serveranwendung bei jedem Systemstart gestartet und leitet die
gesammelten Daten an jeden verbundenen Client weiter.

\subsection{Bibliotheken und Abhängigkeiten}
Als Bibliotheken kommen bei FreeJTAG vor allem Teile der Boost Bibliothek zum
Einsatz. Diese dienen der asynchronen Netzwerkkommunikation (Boost ASIO), dem
hierfür nötigen Einsatz von Threads (Boost Thread), der Verwaltung der
Zeitstempel (Boost Chrono) und dem Speichern von Programmeinstellungen (Boost
Program\_Options).

Außerdem hängt FreeJTAG von dem Paket "`io"' ab, da dieses für die Aktivierung
des \gls{uart}, wie in \autoref{subs:aktuart} beschrieben, zuständig ist und
installiert werden muss. Die in \autoref{subs:aktuart} beschriebene Skriptdatei
ist ebenso in diesem Paket enthalten. 

Durch die Installation dieser Anwendung wird das Carambola also komplett für den
Einsatz als Entwicklungssystem konfiguriert.

\subsection{Strukturierung der Serversoftware}
<Bild!>
\subsection{ASIO - Asynchrone Netzwerkkommunikation}
\begin{itemize}
  \item Integration in OpenWRT
  \item Bibliotheken
  \item Aufbau der Anwendung
  \item ASIO
  \item \ldots
\end{itemize}
\section{Client - The Kraken}
\begin{itemize}
  \item Funktionalität - Verbinden, Trennen, Synchronisieren, Erfassen
  \item Sortierung der Daten - Treemap
  \item MVC, maven und andere Spielereien
\end{itemize}
\section{Deployment}
\begin{itemize}
  \item Scripting von OpenOCD
  \item Einbindung in Eclipse
\end{itemize}
\section{Datenanalyse (FreeJTAG)}
<\ldots>