\chapter{Realisierung}
\section{Konfiguration des Carambola}
OpenWRT bietet mit dem Programm \gls{uci} eine Anwendung,
Einstellungen zentralisiert zu verwalten. Über das \gls{uci}-System lassen sich
unter anderem Ethernet, WLAN, DHCP und SSH-Server konfigurieren.

Nach Systemstart werden alle Dateien im Ordner \texttt{/etc/uci-defaults}
eingelesen, ausgeführt und anschließend gelöscht. Dies erlaubt es
Softwarepaketen, Einstellungen am System vorzunehmen. Werden diese
Softwarepakete bereits in die Firmware integriert, entspricht dies einer Art
Vorkonfiguration des Systems.
\subsection{WLAN und Netzwerkkonfiguration}
Da die Netzwerkkonfiguration von OpenWRT vollständig durch die
\gls{uci}-Konfigurationsdateien abgewickelt wird, muss auch \gls{uci} genutzt
werden, wenn das Netzwerk vorkonfiguriert werden soll.

\gls{uci} bietet mit \listinlsh{uci batch} explizit einen Befehl an, um <\ldots> 
\begin{definition}[Here document]
Ein \emph{Here document} ermöglicht es unter einer Unix Shell, ein normalerweise
interaktives Programm mit definierten Daten zu speisen.
\end{definition}

 <UCI, Konfigurationsskript>
\subsection{Aktivierung des UART}
Der Mikrochip\cite{RA01} des Carambolas besitzt zwei \glspl{uart}. Einer dieser
\glspl{uart} wird für die Serielle Konsole verwendet und somit kann der zweite
Anschluss für die Datenschnittstelle des Entwicklungssystems verwendet werden.
Da die Pins des Carambolas für den zweiten \gls{uart} standardmäßig auf
\gls{gpio}-Betrieb eingestellt sind, müssen diese erst umgestellt werden.

Hierzu muss das Tool \texttt{io} installiert und mittels 
\listinlsh{io 0x10000060 0x01} ausgeführt werden. Dies setzt im Speicher des
Mikrocontrollers das Flag, den \gls{uart} zu aktivieren. Dieser Vorgang muss
nach jedem Systemstart erfolgen und wird 

<Deaktivierung der Konsole, Aktivierung des zweiten UART>
\section{OpenOCD}
\begin{itemize}
  \item Erstellung des Makefiles 
  \item Vorkonfiguration von OpenWRT
\end{itemize}
\section{Server - FreeJTAG}
\begin{itemize}
  \item Integration in OpenWRT
  \item Bibliotheken
  \item ASIO
  \item Aufbau der Anwendung
  \item \ldots
\end{itemize}
\section{Client - The Kraken}
\begin{itemize}
  \item Funktionalität - Verbinden, Trennen, Synchronisieren, Erfassen
  \item Sortierung der Daten - Treemap
  \item MVC, maven und andere Spielereien
\end{itemize}
\section{Deployment}
\begin{itemize}
  \item Scripting von OpenOCD
  \item Einbindung in Eclipse
\end{itemize}
\section{Datenanalyse (FreeJTAG)}
<\ldots>