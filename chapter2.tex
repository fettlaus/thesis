\chapter{Analyse eines Entwicklungssystems}\label{chap:analyse}
\minitoc
In diesem Kapitel soll aufgezeigt werden, woraus ein Entwicklungssystem
im Allgemeinen besteht und welche Anforderungen sich also für ein zu
entwerfendes Entwicklungssystem ergeben.
\section{Begriffsdefinitionen}
\subsection*{Zielsystem}
Das \emph{Zielsystem} ist das zu entwerfende Entwicklungsziel.
\subsection*{Entwicklungsplattform}
Zu einer \emph{Entwicklungsplattform} gehören alle Hardware- und
Softwareelemente, die nötig sind, um die Umsetzung eines bestimmten
Softwareprojektes zu erreichen. Die einzelnen Be                      
 standteile einer solchen \emph{Entwicklungsplattform} sollen im Verlauf dieses
Kapitels identifiziert werden.
\section{Elemente einer Entwicklungsplattform}
Die Elemente einer Entwicklungsplattform kann man drei Kategorien zuordnen.
Jeder Bestandteil 
\subsection{Software}
Text
\begin{itemize}
  \item \subsubsection*{Compiler} GCC
  \item \subsubsection*{Debugger} GDB
  \item \subsubsection*{Testtools} Junit
  \item \subsubsection*{Deploymenttools}
\end{itemize}

\subsection{Hardware}
\begin{itemize}
  \item Hardware-Testtools
  \item Flasher
  \item ns
\end{itemize}
\subsection{Vorgehensmodell}
Am Anfang eines Entwicklungsprozesses steht die Entscheidung über eine
Vorgehensweise. Viele Vorgehensweisen basieren dabei auf Vorgehensmodellen,
die in Industrie und Wirtschaft weit verbreitet sind. Als Beispiele lassen sich
hier das Wasserfallmodell, Scrum oder auch Extreme Programming nennen.

All diesen Modellen ist gemein, dass einige ihrer Schritte unbedingt
Softwareunterstützung bedürfen oder diese die Entwicklung unter
Umständen beschleunigen kann.
Einige Vorgehensmodelle und deren nötige Softwareunterstützung sollen hier
exemplarisch aufgeführt werden.
\subsubsection*{Wasserfallmodell}
Das Wasserfallmodell ist ein klassisches, nicht-iteratives Vorgehensmodell,
dessen Ablauf man sich anhand eines Wasserfalls gut visualisieren kann. 


\begin{figure}[h!]
\centering
\includegraphics[width=350pt]{HAW_logo}
\caption{Ablauf eines Softwareprojektes nach dem Wasserfallmodell}
\end{figure}

Jeder einzelne Schritt in der Entwicklung stellt hierbei eine abgeschlossene
Phase mit vorher festgelegten Anfangs- und Endpunkten dar.

 \subsubsection*{Scrum} Hier Scrum beschreiben
\subsubsection*{Extreme Programming}
Hier Extreme Programming beschreiben
\section{Softwareunterstützung}
Für eine neue Entwicklungsplattform ist es wünschenswert, sich nicht von einem
speziellen Vorgehensmodell abhängig zu machen. Es sollte also eine Schnittmenge
gefunden werden, um die Minimalanforderungen möglichst vieler Modelle abdecken
zu können.

Als Schnittmenge kristallisieren sich dabei folgende Aspekte heraus:
\begin{itemize}
  \item \subsection*{Programmieren} Eine Softwareunterstützung erfolgt in diesem
  Schritt meist durch eine IDE und/oder einen Compiler.
  \item \subsection*{Testen} Je nach Art der Tests wird auch in diesem Schritt
  Softwareunterstützung benötigt. 
  \item \subsection*{Verteilen} Soll eine Software auf mehrere
  Systeme verteilt werden, passiert dies fast immer softwaregestützt.
  \item \subsection*{Warten} Die Wartung einer Software ist ein weiterer
  wichtiger Bestandteil des Softwarelebenszyklus'.
\end{itemize}
\section{Vorbedingungen durch das zu entwickelnde System}\label{sec:vorb}
Entwicklung von Software für funkende Mikrocontroller. Dies Mikrocontroller sind
teilweise hochgradig in ihre Umgebung integriert. So können diese z.B.
nicht für die Entwicklung kurzfristig ausgebaut werden. Aufgrund ihrer extrem
hohen Verbreitung\footnote{Referenz} bezieht sich diese Arbeit ausschließlich
auf Systeme mit ARM-Architektur.
\section{Weitere Gegebenheiten}\label{sec:gegeb}
Es existiert Infrastruktur, die man sich für die Entwicklung zu Nutze machen
kann. Hierzu zählen zum Beispiel WLAN, Stromversorgung, Ethernet.

\section{Randbedingungen des Entwicklungsprozesses}
Die gegebenen Umständen aus \autoref{sec:gegeb} und \autoref{sec:vorb} ergeben
folgende Randbedingungen:
\begin{itemize}
  \item Das Zielsystem ist nie auch gleichzeitig das Entwicklungssystem.
  Zusätzlich zu der Tatsache, dass die Zielsysteme eine andere
  \item Das Zielsystem interagiert mit Umgebung. Deshalb sind zum Beispiel
  Funkstörungen o.Ä. zu beachten
  \item Der Zielsystemort ist selten auch der Entwicklungsort
  \item Es existieren viele Zielsysteme mit gleicher oder ähnlicher
  Konfiguration
\end{itemize}

\section{Anforderungen an das Entwicklungssystem}
Aus den Randbedingungen ergeben sich harte und weiche Anforderungen an ein
potentielles Entwicklungssystem. Die harten Anforderungen müssen von einem
Entwicklungssytem zwangsweise erfüllt werden um als geeignetes System dienen zu
können. Die Erfüllung weicher Anforderungen kann den den
Entwicklungsprozess vereinfachen und dem Entwickler die Arbeit erleichtern.
\subsection{Harte Anforderungen}
\begin{itemize}
  \item Schon während der Entwicklung muss ein Zugriff auf das Zielsystem zum
  Debuggen möglich sein
  \item Software muss "`remote"' deployt werden können
  \item Tests sollten aus Simulationsgründen vor Ort erfolgen
  \item Wartungszyklen und Wartungsupdates müssen meist für viele Systeme
  parallel duchgeführt werden
\end{itemize}
\subsection{Weiche Anforderungen}
\begin{itemize}
  \item 
\end{itemize}


\section{Entwicklungsmodelle}
Unterschiedliche Anforderungen je nach Entwicklungsziel
Manche stimmen jedoch überein, so z.B.
  



Abhängig von den Vorfaben und den eingesetzten Entwicklungsmodellen werden
für jedes Softwareprojekt unterschiedliche Anforderungen an die verwendete Hard-
und Software gestellt.

Diese Anforderungen 

 Je nach Art der Software, Anforderungen und

Unabhängig von der Wahl des Entwicklungsmodells können und
müssen viele dieser Schritte teilweise auch softwaregestützt ablaufen. Es ist
daher für ein Entwicklungssystem zwingend notwendig, diese Schritte zu
unterstützten.

Zu den softwaregestützten Schritten zählen hierbei:
\begin{itemize}
  \item \subsection*{Modultest} Modultest
  \item \subsection*{Integrationstest} Integrationstest eingebetteter Systeme
  erfolgen meist mit Hilfe standardisierter Verfahren. Diese verfahren lassen
  sich oft auch zum Beispiel in die Endkontrolle von herstellungsprozessen
  integrieren. So sind in der Industrie verfahren wie JTAG oder Pintests weit
  verbreitet.
  \item \subsection*{Deployment} Im Bereich der 
  \item \subsection*{Wartung}
\end{itemize}

\ldots
\section{JTAG}
JTAG statt SWD
\begin{itemize}
  \item Wird von viele Prozessoren unterstützt.
  \item Ermöglicht ein umfangreiches Debuggen des Prozessors bis auf
		Registerebene.
  \item OpenOCD hat bereits viele Funktionen.
\end{itemize}
\section{Netzwerkprotokoll}
\subsection{Zeitsynchronisation}
\section{Serversoftware}

\section{Clientsoftware}

\lipsum